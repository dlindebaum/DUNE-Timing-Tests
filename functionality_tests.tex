\documentclass{dune}
\usepackage[utf8]{inputenc}
\usepackage{graphicx}
\usepackage{float}
\usepackage{xspace}

\usepackage{biblatex}
\addbibresource{references.bib}

\input{glossary.tex}

\title{DUNE Timing Functionality Tests}
\author{D. Cussans, D. Lindebaum, D. Newbold, S. Paramesvaran, S. Trilov}
\date{October 2022}

\begin{document}
\linenumbers
\maketitle

\tableofcontents


\section{Introduction}
This document lists and describes tests on the DUNE timing system, which is responsible for distributing timestamps throughout the DUNE experiment, including to the \dword{vd} \dword{tde}, which uses a different timestamp distribution system, \dword{wr}.
The test stands currently being used for the majority of the testing are built with prototype hardware.

The success of ProtoDUNE-I indicates the majority of the timing system works together as required.
We highlight specific tests outside the general running of ProtoDUNE-I below.

\section{Functionality Tests}

\subsection{Modified timing protocol}
The difficulty in sourcing \dword{cdr} (Clock and Data Recovery) chips (ADN2814 chip) required the development of a new protocol which is not dependent upon this chip.
The new protocol is designed to have the same functionality as the previous protocol.
\begin{itemize}
  \item Basic functionality tests have passed by demonstrating the same results with basic commands as the previous protocol. Basic tests include:
  \begin{itemize}
  	\item Clocks are produced and received.
  	\item Endpoints reach required states.
  	\item \dword{hsi} information is correctly stored.
  \end{itemize}
  \item Tests listed below can be assumed to have passed the modified protocol, unless specified.
\end{itemize}

\subsection{Clock jitter}
\label{sec:jitter}
The clock must be stable such that it will not drift over time and produce incorrect timestamps.
\begin{itemize}
  \item Difference in time of clock edges between a master and locked endpoint were measured and found to have a standard deviation of 12ps \cite{jitter}.
\end{itemize}

\subsection{Integration with VD TDE}
The \dword{vd} \dword{tde} use a \dword{wr} timing system.
The \dword{wr} clock and timestamp must be locked to the \dword{dts} clock and timestamp.
The following properties were tested:
\begin{itemize}
  \item The \dword{dts} and \dword{wr} clocks remain locked.
  \begin{itemize}
    \item The difference in rising edges between \dword{dts} and \dword{wr} clocks was measured to be 509ps with a standard deviation of 41ps \cite{wr}.
  \end{itemize}
  \item The \dword{wr} timestamp generation is stable when driven by one pulse per second from \dword{dts}.
  \begin{itemize}
    \item The period of a 1PPS as sampled by a \dword{wr} \dword{tdc} had a standard deviation of 0.1ns \cite{wr}.
  \end{itemize}
  \item The \dword{wr} timestamp remains locked to the \dword{dts} timestamp.
  \begin{itemize}
    \item The difference in \dword{wr} timestamps and \dword{dts} timestamps is uniform between 0ns and 16ns. This comes from \dword{dts} being quantised in 16ns intervals and \dword{wr} having a finer granularity \cite{wr}.
    \item Tests with the new protocol have not yet been performed.
  \end{itemize}
\end{itemize}

\subsection{Endpoint delays}
To compensate for varying fibre lengths and environmental conditions, each DTS endpoint can be programmed with an individual delay.
\begin{itemize}
  \item Two endpoints are connected to the same master device with different length fibres.
  The endpoints output an electrical pulse on the receipt of a particular fixed-length timing command.
  The time difference between the two pulses represents the delay introduced by the longer optical fibre.
  The two pulses are aligned by applying an appropriate delay to the endpoint with the shorter fibre \cite{delays}.
  \item This test has not yet been performed for the new protocol, but round trip measurements imply proper application.
\end{itemize}

\subsection{Error detection in link acquisition and lock}
An error in the clock signal must be detected and flagged.
In the new protocol, anything which does not give perfect reception of data will cause the state of the endpoint to move into error.
\begin{itemize}
  \item Signal loss detection is found to work reliably when tested.
\end{itemize}
%Two flags are used to indicate a \dword{lol} (Loss Of Lock) and \dword{los} (Loss Of Signal).
%An additional \dword{of} (Other Frequency) flag checks the clock frequency is the expected value.
%These flags come from the \dword{pll} and also \dword{sfp} for the \dword{los}, and are interpreted by the \dword{dts} firmware.
%These signals go high when a fault is recognised to monitor the status of the system.
%\begin{itemize}
%  \item \dword{los} is high when there is no signal given to the device.
%  \item \dword{lol} is high when bad data is given to the device.
%  \item \dword{of} is high when the device is given a good clock with a different frequency than expected by the device.
%\end{itemize}

\subsection{Clock distribution}
Endpoint must be able to reconstruct a clock from the data they receive.
\begin{itemize}
  \item \dword{cdr} block inside timing firmware able to recover the clock from data.
  The results of \ref{sec:jitter} demonstrates the clocks lock.
\end{itemize}

\subsection{Controlling endpoint transmission}
The transmission of data through the \dword{sfp} connected to the endpoint can be toggled on and off via a software command.
\begin{itemize}
  \item Power is received in the \dword{sfp} of the connected master when the endpoint transmission is toggled on, and no power is received when the endpoint transmission is toggled off.
  \item This is yet to be tested with all available hardware designs.
\end{itemize}

\subsection{Synchronisation across endpoints}
All devices in the \dword{dts} should have the same timestamps.
\begin{itemize}
  \item ProtoDUNE-I included tracks which caused multiple \dword{hsi} events.
    No unexpected behaviour was observed.
  \item Quantitative tests are expected to be carried out with the new protocol in ProtoDUNE-II.
\end{itemize}

\subsection{Bi-directional communication}
To be able to measure round trip times, and more, it is necessary to have a two-way connection between master and endpoint.
\begin{itemize}
  \item The ability to make delay measurements shows this is true, since it requires sending a package from the master and waiting for a reply package from the endpoint.
  \item Additionally, endpoint registers can been read from the master using an appropriate command.
\end{itemize}

\subsection{Broadcasting commands}
The timing system must be able to send sync commands from a master device to an endpoint.
\begin{itemize}
  \item The software interface to inject a commands at arbitrary times has been tested, and shown to work.
\end{itemize}

\subsection{Data packet transfer}
Endpoints must be able to receive data packets, e.g. for changing the value in one of the registers.
\begin{itemize}
  \item Registers on endpoints can be changed by a command from the master.
\end{itemize}

\subsection{Redundancy}
The \dword{dts} will have two GPS receivers and crates containing master devices.
The timestamps between these systems will be compared and any disparity will be flagged and reported to the \dword{ccm} (Control, Configuration and Management).
\begin{itemize}
  \item The GPS systems have not yet been fully developed.
\end{itemize}

\printglossary
\printbibliography

\end{document}