\documentclass{dune}
\usepackage[utf8]{inputenc}
\usepackage{graphicx}
\graphicspath{ {./figures/} }
\usepackage{float}
\usepackage{xspace}

\usepackage{biblatex}
\addbibresource{references.bib}

\input{glossary.tex}

\title{DUNE Timing Functionality Tests}
\author{D. Cussans, D. Lindebaum, D. Newbold, S. Paramesvaran, S. Trilov}
\date{October 2022}

\begin{document}
\linenumbers
\maketitle

\tableofcontents


\section{Introduction}
This document lists and describes tests on the \dword{dts}, which is responsible for distributing timestamps throughout the DUNE experiment, including to the \dword{vd} \dword{tde}, which uses a different timestamp distribution system, \dword{wr}.
The test stands currently being used for the majority of the testing are built with prototype hardware.

The successful running of ProtoDUNE-I confirmed the baseline functionality of the \dword{dts}.
Tests carried out during ProtoDUNE-I are listed first, followed by additional tests which have been performed after ProtoDUNE-I, covering additional aspects of the timing system which deserve testing.

\section{ProtoDUNE-I tests}
\subsection{Integration tests}
As part of the \dword{dts} prototyping process, a Schroff 11890-170 \dword{utca} crate, and an \dword{afc} board \cite{amc_ohwr} was purchased.
The \dword{afc} is being considered as the \dword{amc} which hosts the \dword{fib}.
The two units were operated together, and the existing timing \dword{fmc} was used to start assessing \dword{afc}’s suitability as a fan-out \dword{amc}.

\subsubsection{$\mu$TCA crate-AFC}
The following operations were successfully tested with the \dword{utca} crate and the \dword{afc}.
\begin{itemize}
    \item Remote programming of \dword{afc} \dword{fpga}.
    \item Configuration of \dword{afc} clock routing circuity using \dword{ipmi} commands.
    \item Configuration of \dword{afc} on-board oscillator using \dword{ipmi} commands.
\end{itemize}

\subsubsection{AFC-Timing FMC}
The timing \dword{fmc}, in endpoint mode, acted as a downstream/upstream timing link via a \dword{sfp} transceiver, and recovered the clock from the data-stream using a \dword{cdr} chip.
The existing timing endpoint firmware was ported over to the AFC, and the following operations were successfully tested:
\begin{itemize}
    \item IPBus communication over Ethernet (routed via the crate controller) with the \dword{afc} \dword{fpga}.
    \item Establishment of stable downlink between timing master and endpoint, i.e. successful transmission of commands and timestamp from master to endpoint.
    \item Establishment of stable uplink between timing master and endpoint, i.e. successful transmission of commands from endpoint to master.
\end{itemize}
Since ProtoDUNE-I, the setup has been expanded to use a \dword{mib} distributing timing information via the back-plane of a \dword{utca} to a \dword{fib} on an \dword{afc} board connected to a downstream \dword{fmc}:
\begin{itemize}
    \item The \dword{mib} can lock to an upstream timing clock.
    \item The \dword{fib} can lock to a \dword{mib} in the same crate through the back-plane without an \dword{sfp} connection.
    \item An \dword{fmc} connected to the \dword{fib} using an \dword{fmc} can lock to the clock, and successfully receives commands issued by the \dword{mib}.
\end{itemize}

\subsection{Functionality tests}
ProtoDUNE-I used multiple endpoints, so was suitable for testing performance across endpoints simultaneously.
\subsubsection{Synchronisation across endpoints}
All devices in the \dword{dts} should have the same timestamps.
\begin{itemize}
  \item ProtoDUNE-I included tracks which caused multiple \dword{hsi} events.
    No unexpected behaviour was observed.
  \item The offset in clock edges between two endpoints was found to have a standard deviation of 0.140 ns.
\end{itemize}

\section{Additional tests}

\subsection{Modified timing protocol}
\label{sec:new-protocol}
The difficulty in sourcing \dword{cdr} (Clock and Data Recovery) chips (the ADN2814 chip) required the development of a new protocol which is not dependent upon this chip.
The new protocol is designed to have the same functionality as the previous protocol.
\begin{itemize}
  \item Basic functionality tests have passed by demonstrating the same results as the previous protocol.
  These tests include:
  \begin{itemize}
  	\item Clocks are produced and received.
  	\item Endpoints reach required states.
  	\item \dword{hsi} information is correctly stored.
  \end{itemize}
\end{itemize}
Tests listed below can be assumed to have passed the modified protocol, unless specified.

\subsection{Broadcasting data}
The \dword{dts} must be able to broadcast data through the \dword{sfp} connections.
Commands are sent with fixed latency and required specification in firmware to have an effect.
Arbitrary data packets may be sent to interact with an endpoint in other ways, i.e. to change the value in a register on an endpoint.

\subsubsection{Commands}
The timing system can broadcast and receive 255 different commands.
\begin{itemize}
  \item Master devices are observed to send commands and record the which commands and how many have been sent.
  \item Endpoint devices are observed to record the corresponding commands and counts when a connected master device sends a command.
  \item Pulses can been seen from the decoder of a connected endpoint when an appropriate command is sent from the master device.
\end{itemize}

\subsubsection{Data packets}
Endpoints must be able to receive data packets, e.g. for changing the value in one of the registers.
\begin{itemize}
  \item Registers on endpoints can be changed by a command from the master.
\end{itemize}

\subsection{Clock jitter}
\label{sec:jitter}
The clock must be stable such that it will not drift over time and produce incorrect timestamps.
\begin{itemize}
  \item Difference in time of clock edges between a master and locked endpoint were measured and found to have a standard deviation of 12ps \cite{jitter}.
\end{itemize}

\subsection{Integration with VD TDE}
The \dword{vd} \dword{tde} use a \dword{wr} timing system.
The \dword{wr} clock and timestamp must be locked to the \dword{dts} clock and timestamp.
The following properties were tested:
\begin{itemize}
  \item The \dword{dts} and \dword{wr} clocks remain locked.
  \begin{itemize}
    \item The difference in rising edges between \dword{dts} and \dword{wr} clocks was measured to be 509ps with a standard deviation of 41ps \cite{wr}.
  \end{itemize}
  \item The \dword{wr} timestamp generation is stable when the \dword{wr} master is driven by a 10MHz clock from the \dword{dts}.
  \begin{itemize}
    \item The period of a one pulse-per-second (1PPS) signal as sampled by a \dword{wr} \dword{tdc} and synchronised by a \dword{dts} driven \dword{wr} grandmaster had a standard deviation of 0.1ns \cite{wr}.
  \end{itemize}
  \item The \dword{wr} timestamp remains locked to the \dword{dts} timestamp when sampled from the same electrical pulse.
  \begin{itemize}
    \item The difference in \dword{wr} timestamps and \dword{dts} timestamps is uniform between 0ns and 16ns. This comes from \dword{dts} being quantised in 16ns intervals and \dword{wr} having a finer granularity \cite{wr}.
    \item Tests with the new protocol have not yet been performed.
  \end{itemize}
\end{itemize}

\subsection{Endpoint delays}
To compensate for varying fibre lengths and environmental conditions, each DTS endpoint can be programmed with an individual delay.
\begin{itemize}
  \item Two endpoints are connected to the same master device with different length fibres.
  The endpoints output an electrical pulse on the receipt of a particular fixed-length timing command.
  The time difference between the two pulses represents the delay introduced by the longer optical fibre.
  The two pulses are aligned by applying an appropriate delay to the endpoint with the shorter fibre \cite{delays}.
  \item This test has not yet been performed for the new protocol, but round trip measurements imply proper application.
\end{itemize}

\subsection{Error detection in link acquisition and lock}
An error in the clock signal must be detected and flagged.
In the new protocol, anything which does not give perfect reception of data will cause the state of the endpoint to move into error.
\begin{itemize}
  \item Signal loss detection is found to work reliably when tested.
\end{itemize}
%Two flags are used to indicate a \dword{lol} (Loss Of Lock) and \dword{los} (Loss Of Signal).
%An additional \dword{of} (Other Frequency) flag checks the clock frequency is the expected value.
%These flags come from the \dword{pll} and also \dword{sfp} for the \dword{los}, and are interpreted by the \dword{dts} firmware.
%These signals go high when a fault is recognised to monitor the status of the system.
%\begin{itemize}
%  \item \dword{los} is high when there is no signal given to the device.
%  \item \dword{lol} is high when bad data is given to the device.
%  \item \dword{of} is high when the device is given a good clock with a different frequency than expected by the device.
%\end{itemize}

% % Covered in AFC-crate FMC section?
% \subsection{Controlling endpoint transmission}
% The transmission of data through the \dword{sfp} connected to the endpoint can be toggled on and off via a software command.
% \begin{itemize}
%   \item Power is received in the \dword{sfp} of the connected master when the endpoint transmission is toggled on, and no power is received when the endpoint transmission is toggled off.
%   \item This is yet to be tested with all available hardware designs.
% \end{itemize}

\subsection{Bi-directional communication}
To be able to measure round trip times, and more, it is necessary to have a two-way connection between master and endpoint.
\begin{itemize}
  \item The ability to make delay measurements shows this is true, since it requires sending a package from the master and waiting for a reply package from the endpoint.
  \item Additionally, endpoint registers can been read from the master using an appropriate command.
\end{itemize}

\subsection{Redundancy}
The \dword{dts} will have two GPS receivers and crates containing master devices.
The timestamps between these systems will be compared and any disparity will be flagged and reported to the \dword{ccm} (Control, Configuration and Management).
\begin{itemize}
  \item The GPS systems have not yet been fully developed.
\end{itemize}


\section{Regression testing}
\label{sec:test-scripts}
To ensure updates do not break previous functionality, testing scripts were developed to run with a specified configuration of hardware, firmware, and software.
The scripts use data from the firmware interpreted by the software, so are most appropriate for spotting deviations from expected behaviour, rather than as conclusive tests in themselves.

\subsection{Testing method}
A configuration file is used to record the hardware, firmware, and software present in the test.
The configuration files are saved and may be copied to ensure consistency between runs when only one aspect is changed (i.e. using a different hardware device with the same firmware and software).

The scripts download and install the firmware as specified in the configuration file.
The scripts run a set of commands using the timing software, with the commands used based on the firmware that has been installed.
The outputs of the timing commands are saved in a log file.
The log file is then parsed by a python script to determine whether the tests were successful or not.
The final log file contains the configuration used in the test, the command outputs, and a summary of the success of the tests.

The user may also create custom testing scripts to be run after the standard tests are completed.

\subsection{Available tests}
Different hardware, firmware, and software lead to different valid tests available.
The known hardwares, firmwares, and softwares are listed below with an special considerations.

Hardware:
\begin{itemize}
    \item PC053a/PC053d/PC053e - \dword{fmc}s: No considerations.
    \item PC059 - board containing eight \dword{sfp} slots: Requires a \dword{mux} to be selected, and cannot read \dword{sfp} data.
    \item \dword{tlu} - contains multiple hardware pulse inputs: Cannot read \dword{sfp} data.
    \item \dword{fib} - board compatible with a \dword{utca} with eight \dword{sfp} slots to distribute downstream: Requires a \dword{mux} to be selected, and cannot read \dword{sfp} data.
    \item \dword{mib} - board compatible with a \dword{utca} to receive an upstream signal: Cannot read \dword{sfp} data.
\end{itemize}

Firmware:
\begin{itemize}
    \item Master - allows generation and distribution of a clock: Available on \dword{fmc} and PC059.
    \item Overlord - master design which can transmits received triggers to endpoints: Available on \dword{fmc} and \dword{tlu}.
    \item Boreas - master design including \dword{hsi} block: Avaiable on \dword{fmc} and \dword{tlu}.
    \item Ouroboros - master design also including internal endpoint: Available on \dword{fmc}, PC059, and \dword{fib}.
    \item Endpoint - Receives a clock and timestamps: Avaiable on \dword{fmc}.
    \item Chronos - Endpoint design including \dword{hsi} block: Available on \dword{fmc}.
    \item \dword{crt} - Endpoint design including \dword{crt} block: Available on \dword{fmc}.
\end{itemize}

Software:
\begin{itemize}
    \item dtsbutler - New protocol developed for use without \dword{cdr} chip (see section \ref{sec:new-protocol}): Overlord and \dword{crt} firmware not avaiable, and limited hardware available.
    \item pdtbutler - Old protocol requiring \dword{cdr} chip: Master firmware not available.
\end{itemize}

In addition, it is possible to use a loopback configuration, in which the master and endpoint are the same device, which requires an \dword{sfp} connection to itself.
This is only available for the ouroboros firmware.

\subsubsection{Testing sequence}
\label{sec:flowchart}
The tests used in the scripts and the conditions to run them are displayed in figures \ref{fig:flowchart-main} and \ref{fig:flowchart-modules} below.
Figure \ref{fig:flowchart-main} shows the main sequence of tests, and figure \ref{fig:flowchart-modules} shows a few series of tests factorised out of the main sequence for brevity.


% \begin{table}[hb]
% \centering
% \small
% %\begin{tabular}{|p{3cm}|c|c|c|c|c|c|c|c|c|c|c|}
% \begin{tabular}{|p{3.5cm}|@{\hskip2pt}c@{\hskip2pt}|@{\hskip2pt}c@{\hskip2pt}|@{\hskip2pt}c@{\hskip2pt}|@{\hskip2pt}c@{\hskip2pt}|@{\hskip2pt}c@{\hskip2pt}|@{\hskip2pt}c@{\hskip2pt}|@{\hskip2pt}c@{\hskip2pt}|@{\hskip2pt}c@{\hskip2pt}|@{\hskip2pt}c@{\hskip2pt}|@{\hskip2pt}c@{\hskip2pt}|@{\hskip2pt}c@{\hskip2pt}|}
%   \hline
%                             & New               & Old               &                   &                   &                   &                   &                   &                   &                   &                   &                   \\
%   Test                      & protocol          & protocol          & Boreas            & Fanout            & Overlord          & Master            & TLU               & PC059             & FMC               & MIB               & FIB               \\
%   \hline
%   Reset setup validation    & \checkmark        & \checkmark        & \checkmark        & \checkmark        & \checkmark        & \checkmark        & \checkmark        & \checkmark        & \checkmark        & \checkmark        & \checkmark        \\
%   \hline
%   PLL frequency             & \checkmark        & \checkmark        & \checkmark        & \checkmark        & \checkmark        & \checkmark        & \checkmark        & \checkmark        & \checkmark        & \checkmark        & \checkmark        \\
%   \hline
%   SFP transmission          & \checkmark        & \checkmark        & \checkmark        & \checkmark        & \checkmark        & \checkmark        & \checkmark        & \checkmark        & \checkmark        & \checkmark        & \checkmark        \\
%   \hline
%   Timestamp broadcasting    & \checkmark        & \checkmark        & \checkmark        & \checkmark        & \checkmark        & \checkmark        & \checkmark        & \checkmark        & \checkmark        & \checkmark        & \checkmark        \\
%   \hline
%   HSI status                & \checkmark        & \checkmark        & \checkmark        & \textbf{$\times$} & \textbf{$\times$} & \textbf{$\times$} & \checkmark        & \checkmark        & \checkmark        & \checkmark        & \checkmark        \\
%   \hline
%   HSI buffer words          & \checkmark        & \checkmark        & \checkmark        & \textbf{$\times$} & \textbf{$\times$} & \textbf{$\times$} & \checkmark        & \checkmark        & \checkmark        & \checkmark        & \checkmark        \\
%   \hline
%   Applying endpoint delays  & \checkmark        & \checkmark        & \checkmark        & \checkmark        & \checkmark        & \checkmark        & \checkmark        & \checkmark        & \checkmark        & \checkmark        & \checkmark        \\
%   \hline
% \end{tabular}
% \medskip
% \caption{Tests present in a testing suite, with indications of the conditions for the test to run. ``New protocol'' and ``Old protocol'' (columns 1 and 2) refer to the firmware built for use with/without a \dword{cdr} chip respectively.}
% \label{tab:tests}
% \end{table}

\begin{figure}[hb!]
\begin{center}
\includegraphics[width=\linewidth]{figures/flowchart-fdr-main.png}
\caption{Flowchart displaying the conditions for the main series of tests from a series of testing scripts.}
\label{fig:flowchart-main}
\end{center}
\end{figure}
\FloatBarrier

\begin{figure}[ht!]
\begin{center}
\includegraphics[height=0.95\textheight]{figures/flowchart-fdr-modules.png}
\caption{Flowchart displaying the conditions for specific test modules from a series of testing scripts.}
\label{fig:flowchart-modules}
\end{center}
\end{figure}
\FloatBarrier

\printglossary
\printbibliography

\end{document}